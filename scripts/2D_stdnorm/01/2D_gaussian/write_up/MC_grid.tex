\documentclass[preprint,showpacs,preprintnumbers,amsmath,amssymb]{revtex4}
\usepackage{graphicx}% Include figure files
\usepackage{dcolumn}% Align table columns on decimal point
\usepackage{bm}% bold math
\newcommand{\be}{\begin{equation}}
\newcommand{\ee}{\end{equation}}
\newcommand{\bx}{\mbox{\bf x}}        
\newcommand{\bU}{\mbox{\bf U}}
\newcommand{\bV}{\mbox{\bf V}}
\newcommand{\bP}{\mbox{\bf P}}
\newcommand{\bW}{\mbox{\bf W}}
\newcommand{\uu}{\mathcal{U}}
\def\att{                   % mark at the margin
    \marginpar[ \hspace*{\fill} \raisebox{-0.2em}{\rule{2mm}{1.2em}} ]
    {\raisebox{-0.2em}{\rule{2mm}{1.2em}} }
        }

\begin{document}

\section*{Equations}
Consider a system of N points, each with a dimensionality d.
Our target distribution ($\bP$, assuming it is normalized) is defined in terms of some function ($\bW$) which is somehow related to the Potential Energy ($\bV$).
\be
\begin{split}
    \bP(\bx)&:=e^{-\bW(\bx)}\\
    \bW(\bx)&=-\ln\left\{\bP(\bx)\right\}\\
    \bx_i&=(x_i^1,\ldots,x_i^d)\\
    \text{N points}&=\sum_{i=1}^N \bx_i
\end{split}
\ee

To generate $\bP(\bx)$, such that that distance between points follows a regular grid (a quasi-crystal) we introduce a quasi Lennard-Jones Pair-Wise Potential, $\bU$.
\be
\begin{split}
    \bU_{ij}(\bx_i,\bx_j):=4\epsilon &\left\{\left[\frac{\sigma_i\left(\bx_i\right)+\sigma_j\left(\bx_j\right)}{|\bx_{ij}|}\right]^{12} - \left[\frac{\sigma_i\left(\bx_i\right)+\sigma_j\left(\bx_j\right)}{|\bx_{ij}|}\right]^6 \right\}\\
    &\quad \quad |\bx_{ij}|=\sum\left(\bx_i-\bx_j\right)^2
\end{split}
\ee

$\epsilon$ is a parameter essentially playing the role of temperature in our system. 

The function $\sigma$ returns a constant, and is essentially the distance between points in the regular grid representing $\bP(\bx)$.
\be
\sigma_i=\sigma(\bx_i) := c\cdot \left[N\cdot P\left(\bx_i\right)\right]^{-1/d}
\ee

Sigma is given scaling with respect to the total number of points and their dimensionality respectively. 
The constant c should be on the order of 1.

To generate $\bP(\bx)$ a Metropolis Monte Carlo algorithm is implemented with an acceptance criteria in terms of a uniformly distributed random number $0 < t < 1$.
\be
t > \exp\left\{\beta \Delta \uu \right\}
\ee
where,
\be
\uu=\sum_{i=1}^N \bV(\bx_i) + \sum_{i,j}\bU_{ij}(\bx_i,\bx_j)
\ee

This is analagous to evaluating 
\be
\frac{\bP(\bx_{trial})}{\bP(\bx_i)}
\ee

\section*{Gaussain Target Distribution}
Consider a specific case, where we define a harmonic potential
\be
\bW:=\bV=\frac{\bx^2}{2}
\ee

Our definition of $\bP(\bx)$ becomes
\be
\bP(\bx):= \left(2\pi\right)^{-d/2}e^{-\frac{\bx^2}{2} }
\ee
Where the pre-factor is a normalization constant for the target distribution $\bP(\bx)$.

\end{document}
